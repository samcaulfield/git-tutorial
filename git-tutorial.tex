\documentclass{beamer}
\usepackage{listings}
\AtBeginSection[]
{
  \begin{frame}
    \frametitle{Table of Contents}
    \tableofcontents[currentsection]
  \end{frame}
}
\usepackage[official]{eurosym}
% Remove the navigation buttons at the bottom right of each slide.
\setbeamertemplate{navigation symbols}{}
% Set bulletpoints to circles.
\setbeamertemplate{itemize items}[circle]


\title{Git Tutorial}
\author{Netsoc}
% Remove the date.
\date{}
\begin{document}
\frame{\titlepage}


\section{Netsoc}
\begin{frame}
\frametitle{Netsoc Services}
Netsoc offers useful services to its members such as
\begin{itemize}
	\pause
	\item our library,
	\pause
	\item IRC,
	\pause
	\item Netroom,
	\pause
	\item tutorials (e.g. admin, programming, web-dev),
	\pause
	\item and web hosting.
\end{itemize}
\pause

The subscription fee is only \EUR{}2.

\pause

Our weekly meetings are on Wednesdays at 1pm in room 5033 (Arts Block).
\end{frame}


\section{Goals of this tutorial}
\begin{frame}
\frametitle{Goals of this tutorial}
\begin{itemize}
\pause
\item Introduce Git.
\pause
\item Teach you the fundamentals of Git.
\pause
\item Slides are at http://sc.netsoc.ie/git-tutorial/git-tutorial.pdf
\end{itemize}
\end{frame}


\section{What is Git?}
\begin{frame}
\frametitle{What is Git?}
\begin{itemize}
\pause
\item Git is a free and open-source distributed version control system.
\pause
\item Git is fast, lightweight, easy to learn and supports multiple workflows.
\pause
\item Git is used in Facebook, Google, Microsoft, Twitter, LinkedIn, and more
(if you care).
\end{itemize}
\end{frame}


\section{Getting Git}
\begin{frame}
\frametitle{Getting Git}
\begin{itemize}
\pause
\item Linux: you probably know already
\pause
\item Windows and Mac: git-scm.com or xcode or something
\end{itemize}
\end{frame}


\section{Configuring Git}
\begin{frame}
\frametitle{Configuring Git}
\begin{itemize}
\pause
\item Tell Git who you are (used for commit messages):
\pause
\item git config -{}-global user.name "Sam Caulfield"
\pause
\item git config -{}-global user.email "sam@samcaulfield.com"
\pause
\item git config -{}-global core.editor \$EDITOR
\pause
\item git config -{}-list
\end{itemize}
\end{frame}


\section{Creating, Cloning, Deleting}
\begin{frame}
\frametitle{Creating, Cloning, Deleting}
\begin{itemize}
\pause
\item Creating a repository:
	\begin{itemize}
	\pause
	\item mkdir MyGreatProject
	\pause
	\item cd MyGreatProject
	\pause
	\item git init
	\end{itemize}
\pause
\item Cloning a repository:
	\begin{itemize}
	\pause
	\item git clone urlOfTheProject
	\end{itemize}
\pause
\item Deleting a repository:
	\begin{itemize}
	\pause
	\item Simply delete the .git directory in the repository directory.
	\end{itemize}
\end{itemize}
\end{frame}


\section{Adding and Removing}
\begin{frame}
\frametitle{Adding and Removing}
\begin{itemize}
\pause
\item View the status of the repository:
	\begin{itemize}
	\pause
	\item git status
	\end{itemize}
\pause
\item Files can be tracked or untracked. Tracked files can modified or
unmodified.
	\begin{itemize}
	\pause
	\item Tracked: Already committed.
	\pause
	\item Untracked: Not committed.
	\pause
	\item Modified: Has uncommitted changes.
	\end{itemize}
\pause
\item To add (stage) some changes:
	\begin{itemize}
	\pause
	\item Make your changes
	\pause
	\item git add changedFile.java
	\end{itemize}
\pause
\item Note that staged changes haven't been sent anywhere yet! Use git rm
stagedFile to unstage.
\pause
\item Staged files exist in the Staging Area, the place where you get everything
ready for a commit. Commits are what actually change the repository! Staging and
unstaging files is easy, commits, not so much!
\end{itemize}
\end{frame}


\section{Committing}
\begin{frame}
\frametitle{Committing}
\begin{itemize}
\pause
\item Once the changes you want to commit have been "git add"ed, run git commit.
You will be prompted for a commit message. Some of your details from the
configuration step will be shown along with your message.
\pause
\item To see the commit messages:
	\begin{itemize}
	\pause
	\item git log
	\end{itemize}
\end{itemize}
\end{frame}


\section{Pulling and Pushing}
\begin{frame}
\frametitle{Pulling and Pushing}
\begin{itemize}
\pause
\item Just running git commit updates your local repository.
\pause
\item Commits can be sent to and received from other machines, e.g. a central
code server.
\pause
\item Makes sense to run a git pull before making your changes so you're working
with the latest code on the server.
\pause
\item git pull origin master
	\begin{itemize}
	\pause
	\item origin is just the memorable name we have given to the remote
machine (which could be an IP - hard to remember)
	\pause
	\item master is the branch we are pulling from the remote machine - more
on this later
	\end{itemize}
\pause
\item git push origin master: same deal
\pause
\item If all goes well these operations won't produce conflicts, but they may.
More on this later.
\end{itemize}
\end{frame}


\section{Remotes}
\begin{frame}
\frametitle{Remotes}
\begin{itemize}
\pause
\item In Git, other computers are called Remotes.
\pause
\item To push and pull to and from other machines, you have to tell Git where
they are.
\pause
\item When you clone a repository, the machine you cloned from is set up as a
remote called "origin" automatically
\pause
\item Editing remotes:
\pause
	\begin{itemize}
	\item git remote add conorsFork urlOfTheFork
	\pause
	\item git remote remove conorsFork
	\end{itemize}
\pause
\item List remotes:
	\begin{itemize}
	\pause
	\item git remote -v
	\end{itemize}
\end{itemize}
\end{frame}


\begin{frame}
\begin{itemize}
\pause
\item Those commands will probably do 90\% of what you need most of the time.
\pause
\item Sometimes, exceptional things might happen that require a few more commands:
\pause
	\begin{itemize}
	\item Merge Conflicts
	\pause
	\item Stashing
	\pause
	\item Branching
	\end{itemize}
\end{itemize}
\end{frame}


\section{Merge Conflicts}
\begin{frame}
\frametitle{Merge Conflicts}
\begin{itemize}
\pause
\item Sounds scary, but really isn't
\pause
\item The only scary thing is that two people in different places are working
on the exact same lines of your project..
\pause
\item Git can detect the conflict, but will ask you to resolve it
\pause
\item Git adds "conflict markers" to the file to show you where the conflict
happened
\end{itemize}
\end{frame}


\section{Stashing}
\begin{frame}
\frametitle{Stashing}
\begin{itemize}
\pause
\item You're working on a cool new feature.. untracked/modified/staged files
everywhere..
\pause
\item Someone tells you about a bug that needs to be fixed ASAP
\pause
\item Where to put all the changed files?
\pause
\item You could copy them to a different directory, fix the bug and then copy
them back..
\pause
\item But why not let Git do this for you?
\pause
\item Git can "stash" the files until you need them again.
\pause
\item When Git does a stash, it puts your changes away safely, you're left with
a clean working directory, can make changes, then unstash the files.
\end{itemize}
\end{frame}


\begin{frame}
\frametitle{Stashing}
\begin{itemize}
\pause
\item The data structure that contains your stashes is actually a stack
\pause
\item So you could be working on fixing a bug when a higher priority bug report
comes in, stash those changes, do the higher priority one, unstash, and so on.
\pause
\item Three commands and you're good to go:
	\begin{itemize}
	\pause
	\item git stash
	\pause
	\item git stash pop
	\pause
	\item git stash list
	\end{itemize}
\end{itemize}
\end{frame}


\section{Branches}
\begin{frame}
\frametitle{Branches}
\begin{itemize}
\pause
\item Interesting reading: http://nvie.com/posts/a-successful-git-branching-model/
\pause
\item You can probably just use stashing to work on one thing at a time, with
the occasional higher priority task coming in.
\pause
\item But for doing multiple things at once, consider branching
\pause
\item It's simple, you branch off the main branch, work on something, then merge
it back into the main branch when you're done.
\pause
\item git checkout -b coolFeature
\pause
\item You can do commits etc. on this new branch.
\pause
\item git branch master
\pause
\item git merge coolFeature
	\begin{itemize}
	\pause
	\item merges the coolFeature branch into the current branch
	\end{itemize}
\end{itemize}
\end{frame}


\section{Demonstration}
\begin{frame}
\frametitle{Demnstration}
\end{frame}


\end{document}

